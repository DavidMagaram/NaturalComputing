\documentclass{article}
\usepackage[english]{babel}
\usepackage[toc,page]{appendix}
\usepackage[letterpaper,top=2cm,bottom=2cm,left=3cm,right=3cm,marginparwidth=1.75cm]{geometry}

\usepackage{amsmath}
\usepackage[colorlinks=true, allcolors=blue]{hyperref}

\title{Anomaly Detection using Negative Selection Algorithms\\[0.3em] \large Report on Assignment 2}
\author{Group 28}

\begin{document}
\maketitle

\section{Introduction}
Your introduction goes here! Simply start writing your document. 

\section{Methodology}

\section{Results}
\section{Discussion}

\section{Conclusion}

\subsection{How to create Sections and Subsections}

\begin{thebibliography}{9}
\end{thebibliography}

\clearpage
\begin{appendices}
\section*{Task 1: Using the Negative Selection Algorithm}
This section contains the questions and answers from "Your task" on page 2 of the second assignment.
\begin{enumerate}
    \item \texttt{Compute the area under the receiver operating characteristic curve (AUC [1][2]) to quantify how well the negative selection algorithm with parameters $n = 10$ and different values of $r$ ($r = 1$ until $r = 9$) discriminates individual English strings from Tagalog strings by using the files english.train for training and english.test as well as tagalog.test for testing. Which value of $r$ belongs to what AUC in figure 1?}\\
\textbf{Plot 1 has an $r$-value of 1, plot 2 has an $r$-value of 7, and plot 3 has an $r$-value of 4.}
    \item \texttt{How does the AUC change when you modify the parameter $r$? Specifically, what behaviour do you observe at $r = 1$ and $r = 9$ and how can you explain this behaviour? Which value of $r$ leads to the best discrimination?}\\
\textbf{When the ncreasing the $r$-value pass 3 “flattens” the AUC curve, making it much worse at discrimination. At $r$-values 1, 8 and 9, the AUC value is close to 0.5, making it almost no better than random guessing.}
\end{enumerate}

    


\end{appendices}

\end{document}